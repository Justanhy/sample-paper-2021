%%% Question 1 - - - - - - - - - - - - - - - - - - - - - - - - - - - - - -
\question This question is about functional programming as a programming paradigm.
\begin{parts}
\part Reduce all of the following Haskell expressions to normal forms. Your answers \emph{must} include all reduction steps. No marks are awarded for answers which just state the normal form.
\begin{subparts}
\subpart[2] \haskellIn{not False && (True || False)} \droppoints
\begin{solution} \small
\begin{verbatim}
   not False && (True || False)
=> True && (True || False)
=> True && True
=> True
\end{verbatim}
\end{solution}
\subpart[2] \haskellIn{filter (\x -> x /= 2 && x /= 6) [1..6]} \droppoints
\begin{solution} \small
\begin{verbatim}
   filter (\x -> x /= 2 && x /= 6) [1..6]
== filter (\x -> x /= 2 && x /= 6) [1,2,3,4,5,6]
=> 1 : filter (\x -> x /= 2 && x /= 6) [2,3,4,5,6]
=> 1 : filter (\x -> x /= 2 && x /= 6) [3,4,5,6]
=> 1 : 3 : filter (\x -> x /= 2 && x /= 6) [4,5,6]
=> 1 : 3 : 4 : filter (\x -> x /= 2 && x /= 6) [5,6]
=> 1 : 3 : 4 : 5 : filter (\x -> x /= 2 && x /= 6) [6]
=> 1 : 3 : 4 : 5 : filter (\x -> x /= 2 && x /= 6) []
=> 1 : 3 : 4 : 5 : []
== [1,3,4,5]
\end{verbatim}
\end{solution}
\subpart[2] \haskellIn{if (odd 6) then [] else (drop 2 [1..5])}  \droppoints
\begin{solution} \small
\begin{verbatim}
   if (odd 6) then [] else (drop 2 [1..5])
== if (odd 6) then [] else (drop 2 [1,2,3,4,5])
=> if False then [] else (drop 2 [1,2,3,4,5])
=> drop 2 [1,2,3,4,5]
=> drop 1 [2,3,4,5]
=> drop 0 [3,4,5]
=> [3,4,5]
\end{verbatim}
\end{solution}
\subpart[2] \haskellIn{fmap toUpper "seal"} \droppoints
\begin{solution} \small
\begin{verbatim}
   fmap toUpper "seal"
== fmap toUpper ['s', 'e', 'a', 'l']
=> toUpper 's' : fmap toUpper ['e', 'a', 'l']
=> 'S' : fmap toUpper ['e', 'a', 'l']
=> 'S' : toUpper 'e' : fmap toUpper ['a', 'l']
=> 'S' : 'E' : fmap toUpper ['a', 'l']
=> 'S' : 'E' : toUpper 'a' : fmap toUpper ['l']
=> 'S' : 'E' : 'A' : fmap toUpper ['l']
=> 'S' : 'E' : 'A' : toUpper 'l' : fmap toUpper []
=> 'S' : 'E' : 'A' : 'L' : fmap toUpper []
=> 'S' : 'E' : 'A' : 'L' : []
== ['S', 'E', 'A', 'L']
== "SEAL"
\end{verbatim}
\end{solution}
\subpart[2] \haskellIn{map (map snd) [[(4,'c')], [(8,'a'), (15,'r')], []]} \droppoints
\begin{solution} \small
\begin{verbatim}
   map (map snd) [[(4,'c')], [(8,'a'), (15,'r')], []]
=> map snd [(4,'c')] : 
   map (map snd) [[(8,'a'), (15,'r')], []]
=> [snd (4,'c')] : 
   map (map snd) [[(8,'a'), (15,'r')], []]
=> ['c'] : 
   map (map snd) [[(8,'a'), (15,'r')], []]
=> ['c'] : map snd [(8,'a'), (15,'r')] : 
   map (map snd) [[]]
=> ['c'] : [snd (8,'a'), snd (15,'r')] : 
   map (map snd) [[]]
=> ['c'] : ['a', 'r'] : map (map snd) [[]]
=> ['c'] : ['a', 'r'] : map snd [] : map (map snd) []
=> ['c'] : ['a', 'r'] : [] : map (map snd) []
=> ['c'] : ['a', 'r'] : [] : []
== [['c'], ['a', 'r'], []]
== ["c","ar",""]
\end{verbatim}
\end{solution}
\end{subparts}

\part[3] Consider the following definition of the list difference operator \haskellIn{(\\)}:
    \begin{minted}{haskell}
(\\) :: Eq a => [a] -> [a] -> [a]
xs \\ []     = xs 
xs \\ (y:ys) = delete y (xs \\ ys)
    \end{minted}
    Define a function \haskellIn{diff} which is equivalent to \haskellIn{(\\)}, but is defined using \haskellIn{foldl} instead of explicit recursion. \droppoints 
    
    \begin{solution}
    \emph{Application.} One possible answer is:
    \begin{minted}{haskell}
diff = foldl (flip delete)
    \end{minted}
    \end{solution}

\begin{subparts}
\subpart[4] \label{part:strict} Trace how \haskellIn{diff "abbc" "bd"} would be evaluated in a language with \emph{call-by-value} semantics. \droppoints
\begin{solution}
\emph{Comprehension.}
\begin{small}
\begin{verbatim}
   diff "abbc" "bd"
=> foldl (flip delete) "abbc" "bd"
=> foldl (flip delete) (flip delete "abbc" 'b') "d"
=> foldl (flip delete) "abc" "d"
=> foldl (flip delete) (flip delete "abc" 'd') ""
=> foldl (flip delete) "abc" ""
=> "abc"
\end{verbatim}
\end{small}
\end{solution}

\subpart[4] \label{part:lazy} Trace how \haskellIn{diff "abbc" "bd"} would be evaluated in a language with \emph{call-by-name} semantics. You should assume that the value of this expression is required by some other part of the program. \droppoints
\begin{solution} \emph{Comprehension.}
\begin{small}
\begin{verbatim}
diff "abbc" "bd"
=> foldl (flip delete) "abbc" "bd"
=> foldl (flip delete) (flip delete "abbc" 'b') "d"
=> foldl (flip delete) (flip delete 
     (flip delete "abbc" 'b') 'd') ""
=> flip delete (flip delete "abbc" 'b') 'd'
=> flip delete "abc" 'd'
=> "abc"
\end{verbatim}
\end{small}
\end{solution}

\subpart[4] Define a closed type family
\vspace*{0.2cm}
\begin{minted}{haskell}
Delete :: Nat -> [Nat] -> [Nat]
\end{minted}
\vspace*{0.2cm}
which removes the first occurrence of a type of kind \haskellIn{Nat} from a type-level list of types of kind \haskellIn{Nat}. For example, \haskellIn{Delete 'Zero ['Succ 'Zero, 'Zero, 'Zero]} should evaluate to \haskellIn{['Succ 'Zero, 'Zero]}. \droppoints
\begin{solution} \emph{Application.}
\begin{minted}{haskell}
type family Delete (a :: Nat) (xs :: [Nat]) :: [Nat] where
    Delete a '[] = '[]
    Delete a (a ': xs) = xs
    Delete a (b ': xs) = b ': (Delete a xs)
\end{minted}
\end{solution}
\end{subparts}
\end{parts}
